\documentclass[11pt]{article}
\usepackage{balance,graphics,setspace,parskip,times}
\usepackage[margin=1in]{geometry}
\usepackage[inline]{enumitem}
\usepackage[tiny,compact]{titlesec}
\usepackage[citestyle=numeric,backend=bibtex]{biblatex}
\usepackage{txfonts,balance,graphics,color,subfiles,appendix}
\usepackage{microtype,nth,cleveref}
\usepackage[normalem]{ulem}
\usepackage[T1]{fontenc}
\usepackage[pdflang={en-US},pdftex]{hyperref}
\usepackage[en-GB]{datetime2}
\usepackage[all]{hypcap}  % Fixes bug in hyperref caption linking
\usepackage[utf8]{inputenc} % for a UTF8 editor only
\setlist{nolistsep}
\bibliography{references}

\definecolor{linkColor}{RGB}{6,125,233}
\hypersetup{%
  bookmarksnumbered,
  pdfstartview={FitH},
  colorlinks,
  citecolor=blue, % black,
  filecolor=blue, % black,
  linkcolor=blue, % black,
  urlcolor=linkColor,
  breaklinks=true,
  hypertexnames=false
}

\begin{document}
Lately I've been thinking about the ways people make money on the internet.
I've spent some time thinking about people on Amazon Mechanical Turk, as well as 
on sites of work like Uber, Lyft, etc...
but there are other platforms online where people work under different conditions,
and I'm always looking for something more fringe,
less explainable by the research we have,
and just more \textbf{interesting}.
I guess this is a long--winded way of justifying
watching lots of YouTube videos lately.

I've seen a few research papers about streaming
(\cite[see][]
{Kaytoue:2012:WMP:2187980.2188259,Hamilton:2014:STF:2611105.2557048,Zhang:2015:CIL:2736084.2736091})
but I'm thinking about something that only kind of intersects with these works.

\citeauthor{Kaytoue:2012:WMP:2187980.2188259}~\cite{Kaytoue:2012:WMP:2187980.2188259}
look at streaming particularly through the lens of
professional \textbf{gamers}, which I parse out as separate from professional streamers
and YouTubers because the latter group
(which I'll call ``performers")
is getting paid for the performance rather than the activity,
if that makes sense.
To put it another way, putting aside things like sponsorships and whatnot,
if people stopped watching gamers stream,
they would still make money from winning,
being part of a professional,
(relatively conventionally) compensated team, etc...

\citeauthor{Hamilton:2014:STF:2611105.2557048}~\cite{Hamilton:2014:STF:2611105.2557048}
get a little closer to what I'm talking about
--- it talks about the social aspects of streaming on Twitch,
accumulating followers, engaging with the community, etc...
but I'd like to get more into
the relationships that form among performers, as well as
the relationships that performers develop with the \textit{platform}
(whether that's YouTube or Twitch or whatever).

To be more direct about what I think I'm interested in,
or to define it in positive terms rather than negative ones,
I'm thinking about the people that make their living \textit{by} streaming
(instead of maybe \textit{while} streaming).
Maybe the second point here is thinking about
the ``\href{https://ali-alkhatib.com/blog/meta-points}{Meta Points}" post I wrote a while back,
and how we're lacking historical framing on the stuff we're seeing today.
Whether that historical framing answers some questions about these things
is still kind of up in the air, but
I think it helped me make sense of gig work and micro--tasks,
and I see a few questions in
online streaming and video productions
that seems kind of open still.

One of the major questions seems to be
wrestling with the power that platforms have over performers, and
how the organizations that operate these platforms should negotiate with
performers that don't necessarily follow the rules.
Street performance art seems like a good analogy for
people that perform through online platforms
(actually, Cracked uses that exact analogy in
\href{https://www.youtube.com/watch?v=_MIPQNa8uhg}{a recent video}!).
So I've been looking through a few pieces that look informative,
leading to\dots
\begin{itemize}
  \item \citetitle{harrison1990drawing} by \citeauthor{harrison1990drawing}~\cite{harrison1990drawing}
  \item \citetitle{cohen1998radical} by \citeauthor{cohen1998radical}~\cite{cohen1998radical}
  \item \citetitle{groemer2016street} by \citeauthor{groemer2016street}~\cite{groemer2016street}
  \item \citetitle{osorio2009teatro} by \citeauthor{osorio2009teatro}~\cite{osorio2009teatro}
  \item \citetitle{mason1992street} by \citeauthor{mason1992street}~\cite{mason1992street}
\end{itemize}

That being said, I'm noticing that these are a little obscure, and
I'm open to any perspectives that might be
more mainstream,
more relevant to the research idea I've laid out,
more recent,
etc\dots

---

This is one post in my long journey to feel out research topics that I think would be interesting.
Maybe you could say that I'm posting them as an open offer to take this idea and run with it.
I'm not really sure what I plan to do with it, but
if this sparked your interest, let me know.
If you have any suggestions of possibly interesting research topics,
please be sure to \href{https://ali-alkhatib.com/contact}{contact me}.

\printbibliography{}
\end{document}